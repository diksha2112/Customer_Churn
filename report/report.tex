\documentclass[9pt,twocolumn,twoside]{../../styles/osajnl}
\usepackage{fancyvrb}
\journal{i524} 

\title{Predicting Customer Churn Using Apache Spark Machine Learning}

\author[1,*]{Yatin Sharma, Diksha Yadav}

\affil[1]{School of Informatics and Computing, Bloomington, IN 47408, U.S.A.}


\affil[*]{Corresponding authors: yatins@indiana.edu, yadavd@iu.edu}

\dates{project-001, \today}

\ociscodes{Prediction, Bigdata, Apache Spark, MLlib, Hadoop, Analytics}

% replace this with your url in github/gitlab
\doi{\url{https://github.com/cloudmesh/sp17-i524/tree/master/project/S17-IR-P016/report/report.pdf}}


\begin{abstract}
	This report presents a reference implementation to the Customer Churn
	Prediction Project that is built using Apache Spark Machine Learning. In this
	report, we discuss the models used for predicting customer churn along with
	their accuracies. We will also discuss the software deployment on Chameleon and
	Jetstream cloud computing environments using Ansible and Cloudmesh Client
	scripts.
		\newline
\end{abstract}

\setboolean{displaycopyright}{true}

\begin{document}

\flushbottom % Makes all text pages the same height

\maketitle % Print the title and abstract box

\tableofcontents % Print the contents section
\maketitle

\section{Introduction}
	One of the most important business metrics is churn rate- describing the rate at which your customers stop doing business with you. Our aim is to build a predictive model which uses data science to predict in advance which customers are at risk of leaving. Such model will allow the company to be proactive and focus on such customers. We used Apache Spark\cite{www-apache-spark} Machine Learning library(ML)\cite{www-Spark-ML} for fitting a predictive model on our dataset. Detailed analysis and modeling was be carried out in Python Programming language. Application deployment was initially done on localhost, followed by deployment on the cloud.


\section{Loading the data}
The Customer_Data.csv containing 21 columns and 7000 rows was used. Each row in the dataset represents an observed customer and each column represents attribute of that customer. Dataset was loaded into Spark DataFrame \cite{Spark-Dataframe} along with the specified schema. After the dataframe was instantiated, it was queried using SQL queries with the help Pyspark DataFrame API.

\section{Feature Engineering}
Every single information we use to represent a customer is called a “feature” and the activity of finding useful features is called “feature engineering.” In the Customer data set the data is labeled with two classes – Yes (Churned) and 0 (Not Churned). The features for each customer consist of:
	\ 






\section{Execution Summary}
The tentative schedule for  this project has been outlined below:
\begin{enumerate}
	\item {March 13-March 19, 2017:} Create virtual machines on Chameleon, FutureSystems
	and Jetstream clouds
	\item {March 13-March 19, 2017:} Deploy Hadoop cluster to the clouds and install the
	required software packages to the clusters and also finalize data.
	\item {March 20-March 26, 2017:} Data Preprocessing and applying transformation to
	extract features from the data.
	\item {March 27-April 09, 2017:} Use MLlib to train and evaluate various machine
	learning algorithms and choose best based on various performance metrics.
	\item {April 10 - April 16, 2017:} Create deployable software packages in Python.
	\item {April 17-April 23, 2017:} Complete Project Report.
\end{enumerate}

\section{Workflow}
The project will make use of the following four components. 
\begin{enumerate}
    \item Apache Spark
    \item Hadoop
    \item Spark MLlib

\end{enumerate}

\section{Deployment}
	We will deploy our application using Ansible\cite{www-ansible} playbook. Deployment
	of Master/slave nodes will be done hadoop/spark distributed cluster environment.
	Different cloud systems that will be used in the project include
	Chameleon,FutureSystems and JetStream.

\section{Benchmarking}
	Performance of the Hadoop/Spark clusters deployed on different clouds will
	compared for benchmarking.

\section{Conclusion}

TBD

\section{Acknowledgement}
TBD

\bibliography{references}
\end{document}
